\section{Discrete-Time Markov Chains}

\subsection{Definition of a Markov Chain}

\begin{defn}{Markov chain}{}
    A Markov chain is a process \( (X_{n} )_{n\geq 0} \) satisfying
    \[
        \mathrm{Pr} \left\{ X_{n + 1} = j \mid  X_{n} = i, X_{n - 1} = i_{n - 1} , \dots , X_0 = i_0 \right\} = \mathrm{Pr} \left\{ X_{n + 1} = j \mid X_{n} = i \right\}       
    \]
    for all \( n\geq 1 \) and \( j, i, i_{n - 1} , i_0 \in \mathcal{S}  \). A Markov chain is said to be \textbf{temporally homogeneous} if there are numbers \( (p_{ij} )_{i,j \in \mathcal{S} } \) such that
    \[
        \mathrm{Pr} \left\{ X_{n + 1} = j \mid X_{n} = i \right\} = p_{ij} 
    \]
    for all \( n\geq 0 \) and all states \( i,j \in \mathcal{S}  \). The numbers \( (p_{ij} )_{i,j \in \mathcal{S} } \) are generically referred to as the  \textbf{transition probabilities} of the Markov chain. 
\end{defn}

\begin{defn}{Transition matrix}{}
\[
    P = \begin{bmatrix}
        p_{00}  & p_{01}  & \dots   \\
         p_{10} &  p_{11} &   \dots \\
         \vdots& \vdots & \ddots  \\ 
    \end{bmatrix}
\] The matrix \( P \) is called the transition matrix. It is a stochastic matrix, which means it is square with non-negative entries, whose rows sum to one. A Markov chain and transition matrix are equivalent representations of each other.



\end{defn}
\subsubsection{The Chapman-Kolmogorov equations}

\begin{idea}{The Chapman-Kolmogorov Equations}{}

    We define \textbf{multi-step transition probabilities}
    \[
        P_{ij}^n \coloneqq \mathrm{Pr} \left\{ X_{n + m} = j \mid X_{m} = i  \right\} , \ \ n,m\geq 0. 
    \]
    The Chapman-Kolmogorov equations give a recursive formula for computing the \( n \)-step transition probabilities. 

    For all \( m,n \geq 0  \) and states \( i,j \), 
    \[
        P_{ij}^{m+ n} = \sum_{k} P_{ik} ^m P_{kj}^n .    
    \]
    In particular,  we have
    \[
        P ^n= \begin{bmatrix}
            P_{00} ^n & P_{01}  ^n& \dots   \\
             P_{10} ^n&  P_{11} ^n&   \dots \\
             \vdots& \vdots & \ddots  \\ 
        \end{bmatrix}
    \] where \( P^n \) is the transition matrix \( P \) raised to the \( n^\text{th}  \) power. Also note \( P^{m+ n} = P^m P^n  \). 
\end{idea}

\begin{defn}{Irreducible}{}
A \textbf{class} of states is a nonempty set of states such that every state in the set can communicate with one another. These classes form an equivalence relation and partition the state space. We say a Markov chain is irreducible if there is only one class. 


\end{defn}

\begin{defn}{Periodicity}{}

    For a state \( i \), define its \textbf{period}
    \[
        d(i) \coloneqq \mathrm{gcd} \left\{ n\geq 1: P_{i i} ^n  > 0\right\} .
    \]
    States with period 1 are called \textbf{aperiodic}. 
    Periodicity is a class property. 
\end{defn}

\begin{idea}{}{}
Define the \textbf{first return time} for state \( j \in \mathcal{S}  \) as 
\[
    T_{j} \coloneqq \inf \left\{ n\geq 1: X_{n} = j \right\}. 
\]

Let \( (X_{n} ) _{n\geq 0} \) be a Markov chain with transition matrix \( P \). Conditioned on the event \( \left\{ T_{j} < \infty  \right\}  \), the process \( (X_{n + T_{j} })_{n\geq 0} \) is a Markov chain with transition matrix \( P \) and starting state \( j \) and is independent of \( X _0, \dots , X_{T_j} \). 
\end{idea}

\begin{idea}{}{}
    A state \( j \) is \textbf{recurrent} if \( \mathrm{Pr} \left\{ T_{j} < \infty  \mid  X _0 = j\right\}   = 1\), and it is called \textbf{transient} if \(  \mathrm{Pr} \left\{ T_{j} < \infty  \mid  X _0 = j\right\}   < 1 \). Recurrence and transience are class properties. 
\end{idea}

\begin{lem}{}{}
State \( i \) is recurrent if and only if \( \sum_{n = 1}^{\infty} P_{ii} ^n = \infty . \) 
\end{lem}

\subsection{Markov Limit Theorems}

\begin{thrm}{Strong Law of Large Numbers for Markov Chains}{}
Define \( N_{j} (n), \ n\geq 1 \), to be the number of transitions into state \( j \), up to and including time \( n \). More precisely, 
\[
    N_{j} (n)\coloneqq \# \left\{ 1\leq k\leq n: X_{k} = j \right\} . 
\]
Also define the expected first return time to be
\[
    \mu _{j j } \coloneqq \mathbb{E} [T_{j} |X_{0}= j ].
\]

Let \( (X_{n} )_{n\geq 0} \) be a Markov chain starting in state \( X_0 = i. \) If \( i \leftrightarrow j \), then
\[
\frac{    N_{j}(n)}{n} \to  \frac{1}{\mu _{j  j}} \ \text{ a.s.}
\] 
\end{thrm}

\begin{cor}{}{}
    For an irreducible Markov chain \( (X_{n} )_{n\geq 0} \),  we have
    \[
        \lim_{n \to \infty} \frac{1}{n}\sum_{k = 1}^n P_{ij}^k = \frac{1}{\mu _{j j }} . 
    \]
\end{cor}


\subsubsection{Stationary distributions and the issue of convergence}

\begin{defn}{Stationary Distribution}{}

    A probability distribution \( (\pi _{j} )_{j \in \mathcal{S}}  \) is a stationary distribution for a Markov chain with transition probability matrix \( P \) if \( \pi _{j}  = \sum_{i} \pi _{i} p _{ij}\) for each \( j \in \mathcal{S}  \). Equivalently in matrix notation, \( \pi  = \pi P\), when \( \pi    \) is considered as a row vector. 

\end{defn}

\begin{defn}{Positive and Null recurrence}{}

    A recurrent state \( j \) is positive recurrent if \( \mu _{jj} < \infty  \), or null recurrent if \( \mu _{jj} = \infty  \). Positive and null recurrence are class properties.

\end{defn}

It is important to note stationary distributions aren't necessarily unique. It is important to note that stationary distribution does not always exist. 



\begin{thrm}{}{}
An irreducible Markov chain satisfies exactly one of the following:
\begin{enumerate}
    \item All states are transient, or all states are null recurrent. In this case, \( \frac{1}{n}\sum_{k = 1}  ^n P_{ij}^k \to 0 \) as \( n \to \infty  \) for all states \( i,j \), and no stationary distribution exists. 
    \item All states are positive recurrent. In this case, a unique stationary distribution exists and is given by \( \pi _{j} = \frac{1}{\mu _{ j j }} = \lim_{n \to \infty} \frac{1}{n}\sum_{k}^n P^k_{ij}  \). 
\end{enumerate}


\end{thrm}

\begin{thrm}{}{}
    Let \( (X_{n} )_{n\geq 0} \) be an irreducible, aperiodic, and positive recurrent Markov chain with stationary distribution \( \pi \). Then
    \[
        \lim_{n \to \infty} \sum_{j} \left\lvert P_{ij}^n - \pi _{j}   \right\rvert = 0 \text{ for all  } i  \in \mathcal{S} . 
    \] In particular, \( P^n_{ij}  = \pi _{j}\) for all \( i, j \). 
    \end{thrm}

\subsection{Reversibility and Spectral Gap}

\begin{defn}{Reversible Markov chain}{}
A Markov chain with transition matrix \( P \) and stationary distribution \( \pi     \) is reversible if the transition probabilities satisfy
\[
    \pi _{i} p_{ij} = \pi _{j} p_{ji}  \text{ for all states } i,j.
\]
In this case, we say \( (P, \pi ) \) is reversible for convenience.

The equations above are called the detailed balance equations. These equations are also sufficient for reversibility. Hence, if there is a probability distribution \( \pi   \) such that the equations are satisfied, then the Markov chain is reversible with stationary distribution \( \pi  \). 
\end{defn}

\subsubsection{Spectral gap and trend to equilibrium}

\begin{defn}{Total variation distance}{}
For probability measures \( \mu , \nu  \) on a measurable space \( (\Omega . \mathcal{F} ) \), we define their total variation distance
\[
    \lVert \mu - \nu  \rVert _{\mathrm{TV} } \coloneqq \sup_{A\in \mathcal{F} } \left\lvert \mu (A) - \nu (A) \right\rvert. 
\]

Also note that when \( \Omega    \) is countable, we have
\[
    \lVert \mu - \nu  \rVert _{\mathrm{TV} } = \frac{1}{2}\sum_{\omega }\left\lvert \mu (\omega ) - \nu (\omega)  \right\rvert  .
\]
\end{defn}

\begin{defn}{Spectral gap}{}
Define \( \mathrm{Var}_{\pi }(f) = Var(f(X))  \) for \( X \sim \pi  \) and \( f:\mathcal{S} \to \mathbb{R}  \). Also define the function \( Pf: \mathcal{S} \to \mathbb{R}  \) via the matrix-vector multiplication
\[
    (Pf)(i) = \sum_{j} p_{ij}f(j) , \ \ i \in \mathcal{S} .
\]

For a reversible Markov Chain \( (P, \pi ), \) define the spectral gap \( \gamma  \coloneqq 1 -\lambda _2 \), where \( \lambda _2 \) is the smallest number satisfying 
\[
    \mathrm{Var} _\pi   (P f) \leq \lambda _2 \mathrm{Var}_\pi  (f) \  \text{ for all } f:\mathcal{S}  \to \mathbb{R} \text{ with } \mathrm{Var}_\pi  (f)< \infty .  
\]
\end{defn}

\begin{thrm}{}{}
If \( (P, \pi ) \) is a reversible Markov chain with spectral gap \( \gamma  \), then 

\[
    \lVert P^n_{i\bullet} - \pi  \rVert ^2_{\mathrm{TV} } \leq \frac{(1 -\gamma )^n}{\pi _{i} } \text{ for each } n\geq 0 \text{ and each state } i.   
\]
Moreover, if the Markov chain is irreducible, aperiodic and has a finite state space, then \( \gamma >0. \) 
\end{thrm}

\begin{defn}{}{}
Let \( \mu ,\pi  \) be probability distributions on state space \( \mathcal{S}  \). We say that \( \mu   \) has density \( h \) with respect to \( \pi  \) (written \( d \mu = h d \pi  \)) if 
\[
    \mu _{j} = h(j)\pi _{j} \ \ \forall j.
\]
\end{defn}

\begin{lem}{}{}
Let \( (P, \pi ) \) be a reversible Markov chain. If \( d \mu  = h d \pi  \), then \( P^n h  \) is the density of \(\mu P^n  \) with respect to \( \pi . \) 
\end{lem}